\documentclass[a4paper,english,12pt]{article}
\usepackage[utf8]{inputenc}

% Oppsett for norsk
\usepackage[norsk]{babel}
\usepackage{times}
\usepackage[T1]{fontenc}
\usepackage{parskip}
\DeclareUnicodeCharacter{00A0}{ }
\newcommand{\strek}{\textthreequartersemdash}

% Andre pakker
\usepackage{oving}
\usepackage{amsmath}
\usepackage{amssymb}
\usepackage{varioref}
\usepackage{subcaption}
\usepackage{units}
\usepackage{todo}

% Double brackets
\usepackage{stmaryrd}
\usepackage[hidelinks]{hyperref}

% Footnote symbols
\usepackage[symbol*]{footmisc}

\newcommand{\Mod}[1]{\ (\text{mod}\ #1)}

% \mathscr
\usepackage{mathrsfs}

\usepackage[super]{nth}


\title{STA510 --- Statistical modelling and simulation}
\subtitle{Exercise Set 1}
\author{Christian Stigen}
\date{UiS, September \nth{21}, 2017}

\begin{document}
\maketitle

\problem{1 (a)}
The probability of getting \textit{any specific} sequence is exactly the same,
because if we list all possible outcomes for $n$ flips, each sequence will
appear exactly once.

It is understood that each flip is stochastic, meaning there is no memory
involved. Each flip is independent of any other flips. That the coin is
unbiased means that the probability for getting heads or tails is exactly
$0.5$, i.e.~the outcome is binary and evenly distributed. In fact, this is
called a \textit{Bernoulli process}, and each flip is a \textit{Bernoulli
trial}.

In mathematical terms, we have that
\begin{align*}
  P(\textrm{HHTHT}) = P(\textrm{THHHT}) = P(\textrm{five flips}) =
    \left(\frac{1}{2}\right)^5 = \frac{1}{32}
\end{align*}

\problem{1 (b)}
\problem{1 (c)}

\problem{2 (a)}
\problem{2 (b)}

\problem{3}

\problem{4 (a)}
\problem{4 (b)}
\problem{4 (c)}

\problem{5 (a)}
To determine $k$, we use the fact that any PDF must have an area of exactly one
(of course without any negative probabilities). Since $f(x) = 0$ for $|x|
> 1$, we only have to solve
\begin{align*}
  \int_{-1}^{1}{k(1-x^2)~dx} &= 1 \\
  k\left[ x - \frac{1}{3}{x^3} \right]_{-1}^{1} =
  k\left( 1 - \frac{1}{3} + 1 - \frac{1}{3} \right) = \frac{4}{3}k &= 1 \\
   k &= \frac{3}{4}
\end{align*}

\problem{5 (b)}
\problem{5 (c)}

\problem{6 (a)}
\problem{6 (b)}
\problem{6 (c)}

\problem{7 (a)}
\problem{7 (b)}

\problem{8}

\problem{9 (a)}
\problem{9 (b)}

\problem{10 (a)}
\problem{10 (b)}
\problem{10 (c)}
\problem{10 (d)}

\problem{11 (a)}

\end{document}
