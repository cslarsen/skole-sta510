\documentclass[a4paper,english,12pt]{article}
\usepackage[utf8]{inputenc}

% Oppsett for norsk
\usepackage[norsk]{babel}
\usepackage{times}
\usepackage[T1]{fontenc}
\usepackage{parskip}
\DeclareUnicodeCharacter{00A0}{ }
\newcommand{\strek}{\textthreequartersemdash}

% Andre pakker
\usepackage{oving}
\usepackage{amsmath}
\usepackage{amssymb}
\usepackage{varioref}
\usepackage{subcaption}
\usepackage{units}
\usepackage{todo}

% Double brackets
\usepackage{stmaryrd}
\usepackage[hidelinks]{hyperref}

% Footnote symbols
\usepackage[symbol*]{footmisc}

\newcommand{\Mod}[1]{\ (\text{mod}\ #1)}

% \mathscr
\usepackage{mathrsfs}

\usepackage[super]{nth}


\title{STA-510 Statistical modelling and simulation}
\subtitle{Mandatory exercise 2}
\author{Christian Stigen}
\date{UiS, October \nth{19}, 2017}

\begin{document}
\maketitle
\section*{\normalsize{How to run the code}}
All the code for this exercise can be found in \texttt{assignment-2.R}. I have
made a single function for each problem. For example, to see the output of
problem 1 (d), execute the function \texttt{problem1d()}.  From the UNIX
command-line, you can run
\begin{verbatim}
Rscript -e 'pdf("problem1b.pdf");
            source("assignment-2.R");
            problem1b()' > problem1b.out
\end{verbatim}
to produce the textual output file \texttt{problem1b.out} and any plots in
\texttt{problem1b.pdf}. All plots and R output was generated automatically at
the same time as this document.

\problem{1 (a)}
\label{problem.1a}
Because
\begin{align*}
  \rho_{ab} &= \frac{\Cov(X_a, X_b)}{\sqrt{\Var(X_a)\Var(X_b)}} =
    \frac{\sigma_{ab}}{\sigma_a\sigma_b} \\
\end{align*}
we have that $\sigma_{ab} = \rho_{ab}\sigma_a\sigma_b$ and $\sigma_{aa} = 1$.
Furthermore, by definition $\Cov(X,X) = \E(X^2) - \E(X)^2 = \Var(X)$,
so that $\rho_{aa} = 1$ and $\sigma_{aa} = \Var(X_a) = \sigma_a^2$.

The \textbf{covariance matrix} is then
\[
  \Sigma\left(\textbf{X}\right) =
    \begin{bmatrix}
      \rho_{11} & \rho_{12} & \rho_{13} \\
      \rho_{21} & \rho_{22} & \rho_{23} \\
      \rho_{31} & \rho_{32} & \rho_{33} \\
    \end{bmatrix}
  =
    \begin{bmatrix}
      900 & -0.8 \cdot 30 \cdot 10 & 0.2 \cdot 30 \cdot 4 \\
      \rho_{12} & 100 & -0.3 \cdot 10 \cdot 4 \\
      \rho_{13} & \rho_{23} & 16 \\
    \end{bmatrix}
  =
    \begin{bmatrix}
       900 & -240 &  24 \\
      -240 &  100 & -12 \\
        24 &  -12 &  16 \\
    \end{bmatrix}
\]
while the \textbf{expectation vector} is
\[
  \bm{\mu} = \left( \mu_1 , \mu_2, \mu_3 \right)^T 
    = \left( 90, 48, 18 \right)^T
\]

\problem{1 (b)}
Output from R using \texttt{rmvnorm} \cite{mvtnorm}:
\VerbatimInput{problem1b.out}
We could also use the function \texttt{pmvnorm} to calculate the first two:
\VerbatimInput{problem1b_alternative.out}

\problem{1 (c)}
A top-tier character demands that \text{both} $X_1$ and $X_2$ are high at the
same time. Therefore, we should have the highest probability of getting a
top-tier character if they are positively correlated, so that a high $X_1$
implies a high $X_2$ and vice-versa.

\paragraph{(i)   $\rho_{12} =  -0.8$, $\rho_{13} = \rho_{23} = 0$}
\[
  \Sigma\left(\textbf{X}\right) =
    \begin{bmatrix}
      \rho_{11} & \rho_{12} & \rho_{13} \\
      \rho_{21} & \rho_{22} & \rho_{23} \\
      \rho_{31} & \rho_{32} & \rho_{33} \\
    \end{bmatrix}
  =
    \begin{bmatrix}
                         900 & -0.8 \cdot 30 \cdot 10 &    0 \\
      -0.8 \cdot 10 \cdot 30 &                    100 &    0 \\
                           0 &                      0 &   16 \\
    \end{bmatrix}
  =
    \begin{bmatrix}
                         900 &                   -240 &    0 \\
                        -240 &                    100 &    0 \\
                           0 &                      0 &   16 \\
    \end{bmatrix}
\]
This scenario should have the \textit{least probability} of a top-tier
character, because $X_1$ and $X_2$ are correlated negatively: When one is high,
the other tends to be low.

\paragraph{(ii)  $\rho_{12} =   0$, $\rho_{13} = \rho_{23} = 0$}
\[
  \Sigma\left(\textbf{X}\right) =
    \begin{bmatrix}
      \rho_{11} & \rho_{12} & \rho_{13} \\
      \rho_{21} & \rho_{22} & \rho_{23} \\
      \rho_{31} & \rho_{32} & \rho_{33} \\
    \end{bmatrix}
  =
    \begin{bmatrix}
                         900 &    0 \cdot 30 \cdot 10 &    0 \\
         0 \cdot 10 \cdot 30 &                    100 &    0 \\
                           0 &                      0 &   16 \\
    \end{bmatrix}
  =
    \begin{bmatrix}
                         900 &                      0 &    0 \\
                           0 &                    100 &    0 \\
                           0 &                      0 &   16 \\
    \end{bmatrix}
\]

\paragraph{(iii) $\rho_{12} = 0.8$, $\rho_{13} = \rho_{23} = 0$}
\[
  \Sigma\left(\textbf{X}\right) =
    \begin{bmatrix}
      \rho_{11} & \rho_{12} & \rho_{13} \\
      \rho_{21} & \rho_{22} & \rho_{23} \\
      \rho_{31} & \rho_{32} & \rho_{33} \\
    \end{bmatrix}
  =
    \begin{bmatrix}
                         900 &  0.8 \cdot 30 \cdot 10 &    0 \\
       0.8 \cdot 10 \cdot 30 &                    100 &    0 \\
                           0 &                      0 &   16 \\
    \end{bmatrix}
  =
    \begin{bmatrix}
                         900 &                    240 &    0 \\
                         240 &                    100 &    0 \\
                           0 &                      0 &   16 \\
    \end{bmatrix}
\]
This scenario should have the \textit{highest probability} of a top-tier
character, because the two values are positively correlated. That is, they both
tend to be either high or low at the same time.

\problem{1 (d)}
Output from R using \texttt{rmvnorm}:
\VerbatimInput{problem1d.out}
As we can see from the output, we seem to have reasoned correctly in problem 1
(c) about which scenarios should have the highest and lowest probability of
producing a top-tier character.

\problem{2 (a)}
The transform method in the lectures seem to first have been presented by
Çinlar in \cite[p.~96]{cinlar}. While I do not have the book, the key theorem 
is reproduced in \cite{generating}. I have quoted that verbatim as
theorem \ref{theorem:cinlar} \vpageref[below]{theorem:cinlar}.

~\begin{theorem}
  \label{theorem:cinlar}
  Let $\Lambda(t), t \geqslant 0$ be a positive-valued, continuous,
  nondecreasing function. Then the random variables $T_1, T_2, \dots$ are event
  times corresponding to a nonhomogeneous Poisson process with expectation
  function $\Lambda(t)$ if and only if $\Lambda(T_1), \Lambda(T_2), \dots$ are
  the event times corresponding to a homogeneous Poisson process with rate one.
\end{theorem}

In other words, it relates a \textit{homogeneous} Poisson process with a
\textit{constant rate of one} to a non-homogeneous Poisson process. We can
actually go backwards by sampling event times from the homogeneous, rate 1
Poisson process, then go backwards through $\Lambda^{-1}$ to then generate
samples from the non-homogeneous process. See algorithm \vref{algorithm:nhpp}.

\begin{algorithm}
  \caption{Generates $n$ numbers for the non-homogeneous Poisson process (NHPP)}
  \label{algorithm:nhpp}
  \begin{algorithmic}[1]
    \Function{rnhpp}{$n, \Lambda^{-1}$}
      \Let{$s$}{vector of size $n$}
      \For{$i \gets 1 \textrm{ to } n$}
        \Let{$w$}{\Call{rpois}{$1$, lambda=$1$}} \Comment{Homogeneous Poisson sampler}
        \Let{$s_i$}{$\Lambda^{-1}(w)$}
      \EndFor
      \State \Return{$s$}
    \EndFunction
  \end{algorithmic}
\end{algorithm}

In algorithm \vref{algorithm:nhpp}, $\Lambda^{-1}$ is the inverted rate
function, while $\textrm{rpois}$ is a function for generating numbers from a
homogeneous Poisson. The algorithm operates in a for-loop. In the
implementation, we will operate on vectors.

In practice, this means that the rate function $\Lambda(t)$ must be readily
reversible. That may not always be the case, but suffices for the current task.
We will start by finding it.

The intensity is given by $\lambda(t) = 14t^{0.4}$. We then have that
\begin{align*}
  \Lambda(t) &= \int_0^t{\lambda(u)}\, \textrm{d}u
    =  14\int_0^t{u^{0.4}}\, \textrm{d}u
    = \frac{14}{1.4}\left[ u^{1.4} \right]_0^t = 10t^{1.4} = 10t^{\frac{7}{5}}
\end{align*}
If we set $w = \Lambda(t)$, its inverse $\Lambda^{-1}(w) = t$ is given by
\begin{align*}
  w &= 10t^{\frac{7}{5}} \\
  w^{\frac{5}{7}} &= 10^{\frac{5}{7}}t \\
      \left( \frac{w}{10} \right)^{\frac{5}{7}} 
      &= 10^{\frac{7}{5}}w^{\frac{5}{7}} = t \\
  \Lambda^{-1}(w) &= t = 10^{\frac{7}{5}}w^{\frac{5}{7}}
\end{align*}

\paragraph{Sampling from the entire period $t \in [0,5]$}
Each $t$, represented as $w$ in algorithm \vref{algorithm:nhpp}, should cover the
entire period from 0 to 5. For example, when we plot the result, it would not
look good if we didn't plot for the entire range 0--5.

This is difficult to achieve in algorithm \vref{algorithm:nhpp}, since it
generates exactly $n$. I wouldn't want to change the algorithm, in fear of
disturbing the true distribution of numbers.

A naive way to correct this would be to simply sample a larger number of
values.

\problem{2 (b)}
See the plot in figure \vref{plot:2b}.

\begin{figure}
  \centering
  \includegraphics[width=\textwidth]{problem2b.pdf}
  \caption{Cumulative failures in the first five years.}
  \label{plot:2b}
\end{figure}

\problem{2 (c)}
\problem{2 (d)}
\problem{2 (e)}
\problem{2 (f)}
\problem{2 (g)}
\problem{2 (h)}
\problem{2 (i)}
\problem{3 (a)}
\problem{3 (b)}
\problem{3 (c)}
\problem{3 (d)}

\clearpage
\bibliographystyle{IEEEtran}
\bibliography{IEEEabrv,assignment-2}

\end{document}
